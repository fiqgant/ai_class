\documentclass[11pt]{article}
\usepackage[margin=1in]{geometry}
\usepackage{amsmath,amssymb,bm}
\usepackage[T1]{fontenc}
\usepackage[utf8]{inputenc}
\setlength{\parskip}{0.65em}
\setlength{\parindent}{0pt}

\title{Linear \& Logistic Regression + Regularisasi L2 (Ridge)\\
Hitungan Manual Sangat Detail}
\author{TRPL}

\begin{document}
\maketitle

\section{Tujuan Dokumen}
Dokumen ini menyajikan hitungan manual langkah demi langkah untuk:
\begin{itemize}
  \item \textbf{Linear Regression}: normal equation, prediksi, MSE, dan Ridge (L2).
  \item \textbf{Logistic Regression}: sigmoid, log-loss, gradien, satu langkah GD.
  \item \textbf{Regularisasi L2}: penambahan penalti pada fungsi objektif dan efeknya.
\end{itemize}
Fokus pada contoh kecil agar semua langkah dapat dihitung dengan kalkulator.

\section{Linear Regression (1 fitur + bias)}
\subsection{Model dan Tujuan}
Model linier untuk satu fitur:
\[
\hat{y} \;=\; w\,x \;+\; b
\]
Dengan menyertakan bias sebagai kolom satuan di matriks desain, parameter kita tulis sebagai vektor:
\[
\bm{w_b} \;=\; \begin{bmatrix} w\\ b \end{bmatrix}.
\]
Untuk $n$ contoh, normal equation:
\[
\bm{\hat{w}_b} \;=\; (X^{\mathrm T}X)^{-1} X^{\mathrm T}\bm{y}.
\]

\subsection{Contoh Data Mini dan Pembentukan Matriks}
Ambil data mini: $(x,y)=\{(1,1),(2,2),(3,2)\}$.
\[
X \;=\; \begin{bmatrix}
1 & 1\\
2 & 1\\
3 & 1
\end{bmatrix},
\quad
\bm{y} \;=\; \begin{bmatrix}
1\\
2\\
2
\end{bmatrix}.
\]
Kolom pertama $X$ adalah fitur $x$, kolom kedua adalah bias $1$.

\subsection{Hitung $X^{\mathrm T}X$ dan $X^{\mathrm T}\bm{y}$}
\[
X^{\mathrm T}X
=
\begin{bmatrix}
1 & 2 & 3\\
1 & 1 & 1
\end{bmatrix}
\begin{bmatrix}
1 & 1\\
2 & 1\\
3 & 1
\end{bmatrix}
=
\begin{bmatrix}
14 & 6\\
6 & 3
\end{bmatrix},
\qquad
X^{\mathrm T}\bm{y}
=
\begin{bmatrix}
1 & 2 & 3\\
1 & 1 & 1
\end{bmatrix}
\begin{bmatrix}
1\\
2\\
2
\end{bmatrix}
=
\begin{bmatrix}
9\\
5
\end{bmatrix}.
\]

\subsection{Invers Matriks $2\times 2$ dan Solusi}
Untuk $A=\begin{bmatrix}a&b\\c&d\end{bmatrix}$ berlaku
\(
A^{-1}=\dfrac{1}{ad-bc}\begin{bmatrix}d&-b\\-c&a\end{bmatrix}.
\)
Dengan $a=14,b=6,c=6,d=3$,
\[
ad-bc \;=\; 14\cdot 3 - 6\cdot 6 \;=\; 42-36 \;=\; 6,
\qquad
(X^{\mathrm T}X)^{-1}
=\frac{1}{6}\begin{bmatrix}3&-6\\-6&14\end{bmatrix}.
\]
Maka
\[
\bm{\hat{w}_b}
=
\frac{1}{6}
\begin{bmatrix}
3 & -6\\
-6 & 14
\end{bmatrix}
\begin{bmatrix}
9\\
5
\end{bmatrix}
=
\frac{1}{6}
\begin{bmatrix}
27-30\\
-54+70
\end{bmatrix}
=
\frac{1}{6}
\begin{bmatrix}
-3\\
16
\end{bmatrix}
=
\begin{bmatrix}
-0.5\\
2.6667
\end{bmatrix}.
\]
Jadi $\hat{w}=-0.5$ dan $\hat{b}\approx 2.6667$.

\subsection{Prediksi dan MSE (cek manual)}
Prediksi untuk $x=1,2,3$:
\[
\hat{y}(1)= -0.5\cdot 1 + 2.6667 = 2.1667,\quad
\hat{y}(2)= -0.5\cdot 2 + 2.6667 = 1.6667,\quad
\hat{y}(3)= -0.5\cdot 3 + 2.6667 = 1.1667.
\]
Residual $e_i=y_i-\hat{y}_i$:
\[
\{-1.1667,\; 0.3333,\; 0.8333\}.
\]
MSE:
\[
\mathrm{MSE}
=
\frac{(-1.1667)^2+(0.3333)^2+(0.8333)^2}{3}
\;\approx\;
\frac{1.3611+0.1111+0.6944}{3}
\;=\; \frac{2.1666}{3}
\;\approx\; 0.7222.
\]

\section{Ridge Regression (L2) — Solusi Tertutup}
Ridge menambahkan penalti $\lambda\lVert \bm{w_b}\rVert_2^2$ sehingga solusi menjadi
\[
\bm{\hat{w}_b}^{(\mathrm{ridge})}
=
\big(X^{\mathrm T}X+\lambda I\big)^{-1}X^{\mathrm T}\bm{y}.
\]
Ambil $\lambda=1$:
\[
X^{\mathrm T}X+\lambda I
=
\begin{bmatrix}
15 & 6\\
6 & 4
\end{bmatrix},
\qquad
\det = 15\cdot 4 - 6\cdot 6 = 24,
\]
\[
\big(X^{\mathrm T}X+\lambda I\big)^{-1}
=
\frac{1}{24}
\begin{bmatrix}
4 & -6\\
-6 & 15
\end{bmatrix}.
\]
Sehingga
\[
\bm{\hat{w}_b}^{(\mathrm{ridge})}
=
\frac{1}{24}
\begin{bmatrix}
4 & -6\\
-6 & 15
\end{bmatrix}
\begin{bmatrix}
9\\
5
\end{bmatrix}
=
\frac{1}{24}
\begin{bmatrix}
36-30\\
-54+75
\end{bmatrix}
=
\frac{1}{24}
\begin{bmatrix}
6\\
21
\end{bmatrix}
=
\begin{bmatrix}
0.25\\
0.875
\end{bmatrix}.
\]
Terlihat bobot menjadi lebih kecil (shrinkage), membantu stabilitas saat kolinearitas atau data terbatas.

\section{Logistic Regression (biner)}
\subsection{Model, Sigmoid, dan Log-Loss}
Probabilitas kelas $1$:
\[
p \;=\; \sigma(z) \;=\; \frac{1}{1+e^{-z}},
\qquad
z \;=\; w\,x + b.
\]
Log-loss rata-rata:
\[
J(w,b)
=
\frac{1}{n}\sum_{i=1}^{n}
\Big(-y_i\ln p_i - (1-y_i)\ln(1-p_i)\Big).
\]
Gradien (tanpa reguler):
\[
\frac{\partial J}{\partial w}
=
\frac{1}{n}\sum_{i=1}^n (p_i-y_i)\,x_i,
\qquad
\frac{\partial J}{\partial b}
=
\frac{1}{n}\sum_{i=1}^n (p_i-y_i).
\]

\subsection{Contoh Kecil dan Satu Langkah Gradient Descent}
Ambil data:
\[
(x,y)=\{(0,0),(1,0),(2,1),(3,1)\},
\qquad
w=1,\; b=-1.
\]
Hitung $z_i=wx_i+b$ dan $p_i=\sigma(z_i)$:
\[
\begin{array}{c|c|c|c}
i & x_i & z_i & p_i\\\hline
1 & 0 & -1 & \frac{1}{1+e^{1}} \approx 0.2689\\
2 & 1 & 0 & \frac{1}{1+e^{0}} = 0.5\\
3 & 2 & 1 & \frac{1}{1+e^{-1}} \approx 0.7311\\
4 & 3 & 2 & \frac{1}{1+e^{-2}} \approx 0.8808
\end{array}
\]
Loss per contoh $\ell_i$:
\[
\ell_i
=
- \big(y_i\ln p_i + (1-y_i)\ln(1-p_i)\big).
\]
Rata-rata $J$ (pembulatan):
\[
J
=
\frac{0.3133+0.6931+0.3133+0.1269}{4}
\approx 0.3617.
\]
Hitung selisih $(p_i-y_i)$:
\[
\{0.2689,\; 0.5,\; -0.2689,\; -0.1192\}.
\]
Gradien:
\[
\frac{\partial J}{\partial w}
=
\frac{1}{4}\Big(
0.2689\cdot 0 + 0.5\cdot 1 - 0.2689\cdot 2 - 0.1192\cdot 3
\Big)
=
\frac{-0.3954}{4}
=
-0.0989,
\]
\[
\frac{\partial J}{\partial b}
=
\frac{1}{4}(0.2689+0.5-0.2689-0.1192)
=
\frac{0.3808}{4}
=
0.0952.
\]
Satu langkah GD dengan laju belajar $\eta=0.5$:
\[
w_{\mathrm{baru}}
=
w - \eta \frac{\partial J}{\partial w}
=
1 - 0.5\cdot (-0.0989)
=
1.04945,
\]
\[
b_{\mathrm{baru}}
=
b - \eta \frac{\partial J}{\partial b}
=
-1 - 0.5\cdot 0.0952
=
-1.0476.
\]
Ulangi perhitungan untuk melihat $J$ menurun setelah update.

\section{Regularisasi L2 (Ridge) pada Logistic}
Tambahkan penalti kuadrat bobot:
\[
J_{\mathrm{reg}}
=
J + \lambda\, w^2
\quad\Rightarrow\quad
\frac{\partial J_{\mathrm{reg}}}{\partial w}
=
\frac{1}{n}\sum_{i=1}^{n}(p_i-y_i)\,x_i
\;+\;
2\lambda w.
\]
(Umumnya bias tidak direguler.)

Lanjutkan contoh sebelumnya, ambil $\lambda=0.1$:
\[
\frac{\partial J_{\mathrm{reg}}}{\partial w}
=
-0.0989 + 2\cdot 0.1 \cdot 1
=
0.1011.
\]
Dengan $\eta=0.5$:
\[
w_{\mathrm{baru}}
=
1 - 0.5\cdot 0.1011
=
0.94945,
\qquad
b_{\mathrm{baru}}
=
-1 - 0.5\cdot 0.0952
=
-1.0476.
\]
Efek L2: mendorong $|w|$ mengecil (lebih konservatif), mengurangi overfitting.

\section{Ringkasan Intuisi}
\begin{itemize}
  \item \textbf{Linear}: solusi normal memberikan $\hat{w},\hat{b}$; \textbf{Ridge} menstabilkan solusi saat kolinearitas/fitur banyak dengan mengecilkan bobot.
  \item \textbf{Logistic}: memodelkan probabilitas via sigmoid; optimasi numerik memakai gradien. \textbf{L2} menambah term $2\lambda w$ pada gradien bobot.
  \item Regularisasi mengurangi variance dan sering meningkatkan generalisasi, khususnya untuk data kecil/berisik.
\end{itemize}

\section{Latihan Manual}
\begin{enumerate}
  \item \textbf{Linear}: ganti data menjadi $(x,y)=\{(1,1),(2,2),(4,3)\}$, hitung $X^{\mathrm T}X$, $X^{\mathrm T}\bm{y}$, invers, $\hat{w},\hat{b}$, dan MSE.
  \item \textbf{Ridge}: pada data awal, coba $\lambda\in\{0.5,\,5\}$, bandingkan $\hat{w},\hat{b}$.
  \item \textbf{Logistic}: gunakan $w=0.5,b=-0.5$, hitung $z_i,p_i,\ell_i$, rata-rata $J$, lalu satu langkah GD dengan $\eta=0.2$ (tanpa reguler dan dengan L2 $\lambda=0.1$).
  \item \textbf{Threshold}: gunakan probabilitas logistic, coba ambang $0.4$ dan $0.6$; hitung TP, FP, TN, FN secara manual.
\end{enumerate}

\end{document}
